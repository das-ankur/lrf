\section{Introduction} \label{sec: introduction}

Lossy image compression involves reducing the storage size of digital images by permanently discarding some image data that are redundant or less perceptible to the human eye. This is crucial for efficiently storing and transmitting images, particularly in applications where bandwidth or storage resources are limited, such as web browsing, streaming, and mobile platforms. Lossy image compression methods enable adjusting the degree of compression, providing a selectable tradeoff between storage size and image quality. Widely used methods such as JPEG \cite{wallace1991jpeg} and JPEG 2000 \cite{skodras2001jpeg} follow the \emph{transform coding} paradigm \cite{goyal2001theoretical}. They use orthogonal linear transformations, such as discrete cosine transform (DCT) \cite{ahmed1974discrete} and discrete wavelet transform (DWT) \cite{antonini1992image}, to decorrelate small image blocks. Since these transforms map image data into a continuous domain, quantization is necessary before coding into bytes. However, as quantization errors can significantly degrade compression performance, the quantizers must be carefully crafted to minimize this impact, which further complicates codec design.

Another promising paradigm relies on low-rank approximation methods. Particularly singular value decomposition (SVD) is a representative method known to be the deterministically optimal transform for energy compaction \cite{andrews1976singular}. In practice, current SVD-based methods \cite{andrews1976singular, prasantha2007image, hou2015sparse} can represent image data only with components containing floating-point elements, which necessitates a quantization layer prior to any byte-level processing. However, their compression performance is suboptimal due to higher sensitivity to quantization errors compared to transform-based methods like JPEG, especially at low bit rates.

Motivated by this, we introduce a new variant of integer matrix factorization (IMF), and based on that, develop an effective \emph{quantization-free} lossy image compression method. Our IMF formulation provides a low-rank representation of the image data as a product of two smaller factor matrices with \emph{bounded integer} elements. Since we can directly store and losslessly process these integer matrices at the byte level, quantization is no longer needed, making IMF arguably better suited than SVD for image compression. Another advantage of IMF is that the reshaped factor matrices can be treated as 8-bit grayscale images, allowing any lossless image compression standard to be seamlessly integrated into the proposed framework. We propose an efficient iterative algorithm for IMF using a block coordinate descent (BCD) scheme, where each column of a factor matrix is taken as a block and updated one at a time using a closed-form solution.
 
Our contributions are summarized as follows. We propose a new IMF formulation that enables \emph{quantization-free} image compression. We introduce an efficient algorithm for the IMF problem and prove its convergence. Additionally, we present a powerful and efficient lossy image coder that outperforms JPEG and SVD at low bit rates. To the best of our knowledge, this work is the first effort to explore IMF for image compression. Our method narrows the gap between factorization and quantization by integrating them into a single layer and optimizing the compression system.

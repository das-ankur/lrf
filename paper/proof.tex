\section{Proof of Theorem \ref{thm:convergence}} \label{app: proof}

\begin{theorem}[Global convergence]
    Let $\seq{\bm U^k}[k\in\N]$ and $\seq{\bm V^k}[k\in\N]$ be sequences generated by the proposed Algorithm \ref{alg: bcd for imf}. Then both sequences are convergent to a locally optimal point of the optimization problem \eqref{eq: imf problem}.
\end{theorem}

\begin{proof}
    To study the convergence of the proposed Algorithm \ref{alg: bcd for imf}, we recast the optimization problem \eqref{eq: imf problem} to the following equivalent problem:
    \begin{align}
        \begin{split}
            \minimize_{\bm U \in \R^{M\times R}, \bm V \in \R^{N\times R}} \Psi(\bm U, \bm V) &\coloneqq H(\bm U, \bm V) + f(\bm U) + g(\bm V),\\
            \text{where} \quad~~~ H(\bm U, \bm V) &\coloneqq \|\bm X - \bm U \bm V^{\rm T} \|_{\rm F}^2,\\
            f(\bm U) &\coloneqq \delta_{[a,b]}(\bm U) + \delta_\Z(\bm U),\\
            g(\bm V) &\coloneqq \delta_{[a,b]}(\bm V) + \delta_\Z(\bm V),
        \end{split}
        \label{eq:IMF_surrogate}
    \end{align}
    with $\delta_\mathcal{B}(\cdot)$ as the indicator function of the nonempty set $\mathcal{B}$ where $\delta_\mathcal{B}(\bm x) = 0$ if $\bm x \in \mathcal{B}$ and $\delta_\mathcal{B}(\bm x) = +\infty$, otherwise. By the definition of functions above, it is easy to confirm that the problems \eqref{eq: imf problem} and \eqref{eq:IMF_surrogate} are equivalent.
    
    The unconstrained optimization problem \eqref{eq:IMF_surrogate} consists of the sum of a differentiable (smooth), convex function $H$ with nonsmooth, nonconvex functions $f$ and $g$. This problem has been extensively studied in the literature under the class of unconstrained nonconvex nonsmooth minimization problems.
    One of the common algorithms applied to such a problem class is the well-known forward-backward-splitting (FBS) algorithm [].
    In Algorithm \ref{alg: bcd for imf}, the variables $\bm U$ and $\bm V$ are updated sequentially following block coordinate (BC) descent minimization algorithms, also often called Gauss-Seidel updates or alternating minimization.
    Hence, in this convergence study, we are interested in algorithms that allow BC-type updates for the nonconvex nonsmooth problem of \eqref{eq:IMF_surrogate} []. Specifically, we focus on the proximal alternating linearized minimization (PALM) algorithm \cite{bolte2014proximal}, to relate its convergence behavior to that of Algorithm \ref{alg: bcd for imf}.
    To that end, we show that the updates of Algorithm \ref{alg: bcd for imf} are equivalent to the updates of PALM on the recast problem of \eqref{eq:IMF_surrogate}, and all the assumptions necessary for the convergence of PALM are satisfied by our problem setting.

    The PALM algorithm can be summarized as follows:
    \begin{enumerate}
        \item Initialize $\bm U^0 \in \R^{M\times R}$, $\bm V^0 \in \R^{N\times R}$ 
        \item For each iteration $k=0,1,...$ 
        \begin{align}
            \begin{split}
                \mysubnumber\quad \bm U^{k+1} &\in \prox^f_{c_k} \left(\bm U^k - \frac{1}{c_k} \nabla_{\bm U} H(\bm U^k, \bm V^k)\right), ~\text{with}~ c_k > L_1(\bm V^k)\\
                \mysubnumber\quad \bm V^{k+1} &\in \prox^g_{d_k} \left(\bm V^k - \frac{1}{d_k} \nabla_{\bm V} H(\bm U^{k+1}, \bm V^k)\right), ~\text{with}~ d_k > L_2(\bm U^{k+1})
            \end{split}
            \label{eq:palm_updates}
        \end{align}
    \end{enumerate}
    where the proximal map for an extended proper lower semicontinuous (nonsmooth) function $\func{\varphi}{\R^n}{(-\infty,+\infty]}$ and $\gamma > 0$ is defined as $\prox^\varphi_\gamma(\bm x) \coloneqq \argmin_{\bm w\in\R^n}\left\{\varphi(\bm w) + \frac{\gamma}{2} \|\bm w - \bm x\|^2_2\right\}$, $\nabla_x$ is the partial derivative with respect to $x$, and $L_1 > 0$, $L_2 > 0$ are local Lipschitz moduli, defined in the following proposition.
    It is also remarked that the iterates in \eqref{eq:palm_updates} can be extended to more than two blocks, which is our case in Algorithm \ref{alg: bcd for imf} with a block representing a column of $\bm U$ or $\bm V$, without violation of the convergence. However, for the sake of presentation, we study these BC-type iterates with two blocks of the form \eqref{eq:palm_updates}. 

    The following proposition investigates the necessary assumptions (cf. \cite[Asm. 1 and Asm. 2]{bolte2014proximal}) for convergence iterates in \eqref{eq:palm_updates}.
    \begin{prop}[Meeting required assumptions]\label{prop:assumptions}
        The assumptions necessary for the convergence of iterates in \eqref{eq:palm_updates} are satisfied by the functions involved in the problem \eqref{eq:IMF_surrogate}, specifically:
        \begin{enumerate}
            \item The indicator functions $\delta_{[a,b]}$ and $\delta_\Z$ are proper and lower semicontinuous functions, so do the functions $f$ and $g$;
            \item For any fixed $\bm V$, the partial gradient $\nabla_{\bm U} H(\bm U, \bm V)$ is globally Lipschitz continuous with modulus $L_1(\bm V) = \|\bm V^T \bm V\|_{\rm F}$ defined by
            \begin{equation*}
                \|\nabla_{\bm U} H(\bm U_1, \bm V) - \nabla_{\bm U} H(\bm U_2, \bm V)\| \leq L_1(\bm V) \|\bm U_1 - \bm U_2\|, \quad \forall \bm U_1,\bm U_2 \in \R^{M\times R},
            \end{equation*}
            where $\|\cdot\|$ in this section denotes the $\ell_2$-norm of the vectorized input with the proper dimension (here, with the input in $\R^{MR\times 1}$).
            The similar Lipschitz continuity is evident for $\nabla_{\bm V} H(\bm U, \bm V)$ as well with modulus $L_2(\bm U) = \|\bm U \bm U^T\|_{\rm F}$.
            \item The sequences $\bm U^k$ and $\bm V^k$ are bounded due to the indicator functions $\delta_{[a,b]}$ with bounded $a$ and $b$. Hence the moduli $L_1(\bm V^k)$ and $L_2(\bm U^k)$ are bounded from below and from above for all $k\in\N$.
            \item The function $H$ is twice differentiable, hence, its full gradient $\nabla H(\bm U,\bm V)$ is Lipschitz continuous on the bounded set $\bm U \in [a,b]^{M\times R}$, $\bm V \in [a,b]^{N\times R}$. Namely, with $M > 0$:
            \begin{align}
                \begin{split}
                    \|(\nabla_{\bm U} H(\bm U_1, \bm V_1) - \nabla_{\bm U} H(\bm U_2, \bm V_2), \nabla_{\bm V} H(\bm U_1, \bm V_1) - &\nabla_{\bm V} H(\bm U_2, \bm V_2))\| \\
                    &\leq M \|(\bm U_1 - \bm U_2, \bm V_1 - \bm V_2)\|,
                \end{split}
            \end{align}
            where $(\cdot,\cdot)$ denotes the concatination of the two arguments.
            \item The sets $[a,b]$ and integer numbers are semi-algebraic; so are their indicator functions. The function $H$ is also polynomial, hence it is semi-algebraic. The sum of these functions results in a semi-algebraic function $\Psi$ in \eqref{eq:IMF_surrogate}, hence $\Psi$ is a Kurdyka-Łojasiewicz (KL) function.
        \end{enumerate}
    \end{prop}
    By Proposition \ref{prop:assumptions}, the optimization problem \eqref{eq:IMF_surrogate} can be solved by the BC iterates in \eqref{eq:palm_updates}, due to the following proposition:
    \begin{prop}[Global convergence \cite{bolte2014proximal}]\label{prop:convergence}
        With the assumptions in proposition \ref{prop:assumptions} being met by the problem \eqref{eq:IMF_surrogate}, let $\seq{(\bm U^k, \bm V^k)}$ be a sequence generated by Algorithm \ref{alg: bcd for imf}. Then the sequence converges to a critical point $(\bm U^\star, \bm V^\star)$ of the problem \eqref{eq:IMF_surrogate}, where $0 \in \partial \Psi(\bm U^\star, \bm V^\star)$, with $\partial$ as the subdifferential of $\Psi$.
    \end{prop}

    In the following, we highlight that the iterates in \eqref{eq:palm_updates} can be implemented more simply and more efficiently by Algorithm \ref{alg: bcd for imf} for the problem of image compression. It is noted that the so-called \emph{forward} steps $\bm U^k - \frac{1}{c_k} \nabla_{\bm U} H(\bm U^k, \bm V^k)$ and $\bm V^k - \frac{1}{d_k} \nabla_{\bm V} H(\bm U^{k+1}, \bm V^k)$ in the $\prox$ operators can be replaced by the simple closed-form solutions $\nicefrac{\bm{E}_r \bm{v}_r}{\lVert \bm{v}_r \rVert^2}$ and $\nicefrac{\bm{E}_r^\mathsf{T} \bm{u}_r}{\lVert \bm{u}_r \rVert^2}$ presented in \eqref{eq: bcd subproblem solution: u} and \eqref{eq: bcd subproblem solution: v}, respectively (in the case where the iterates \eqref{eq:palm_updates} are extended to multi-block updates, each block representing one column). This is thanks to the special form of the functions $H(\cdot, \bm V^k)$ and $H(\bm U^{k+1}, \cdot)$ being quadratic functions, each having a global optimal point which ensures a descent in each forward step. 
    Furthermore, the proximal operators $\prox^f_{c_k}$ and $\prox^g_{d_k}$ can efficiently be implemented by the operators $\round$ and $\clamp_{[\alpha,\beta]}$ in \eqref{eq: bcd subproblem solution: u} and \eqref{eq: bcd subproblem solution: v}. The equivalence of these steps is proven in the following lemma.

    \begin{lem}[$\prox$ implementation]
        Consider the operators $\round$ and $\clamp_{[\alpha,\beta]}$ defined in \eqref{eq: bcd subproblem solution: u} and \eqref{eq: bcd subproblem solution: v}.
        % Define the following (elementwise) operators $\func{P_{[a,b]}}{\R}{\R}$, and $\func{T_\Z}{\R}{\R}$:
        % \begin{align}
        %     P_{[a,b]}(x) &\coloneqq \min\{\max\{a, x\}, b\},\\
        %     T_\Z(x) &\coloneqq \max\{\lfloor x \rfloor, \lfloor x+0.5 \rfloor\}. \label{eq:Tz}
        % \end{align}
        Then $\prox^f_{c_k}(\bm W) = \round(\clamp_{[\alpha,\beta]}(\bm W))$ and $\prox^f_{d_k}(\bm Z) = \round(\clamp_{[\alpha,\beta]}(\bm Z))$ for any $\bm W\in \R^{M\times R}$, $\bm Z\in\R^{N\times R}$, and $\round(\clamp_{[\alpha,\beta]}(\cdot))$ being an elementwise operator on the input matrices.
    \end{lem}
    \begin{proof}
        Define the following norms for a given matrix $\bm W \in \R^{M\times R}$:
        \begin{align*}
            \|\bm W\|_{[a,b]} \coloneqq \sum_{i,j \mid a \leq \bm W_{ij} \leq b} \bm W_{ij}^2, \quad 
            \|\bm W\|_a \coloneqq \sum_{i,j \mid \bm W_{ij} < a} \bm W_{ij}^2, \quad 
            \|\bm W\|_b \coloneqq \sum_{i,j \mid \bm W_{ij} > b} \bm W_{ij}^2.
        \end{align*}
        Moreover, note that the $\round$ operator  can be equivalently driven by the following proximal operator:
        \begin{equation}
            \round(\bm W) = \argmin_{\bm U \in \Z^{M\times R}}\{\|\bm U - \bm W\|^2_F\}.
            \label{eq:equivalence_prox_Tz}
        \end{equation}
        The proximal operator can $\prox^f_{c_k}(\bm W)$ can be rewritten as
        \begin{align*}
            \prox^f_{c_k}(\bm W) &= \argmin_{\bm U \in \R^{M\times R}}\{\delta_{[a,b]}(\bm U) + \delta_{\Z}(\bm U) + \frac{c_k}{2} \|\bm U - \bm W\|^2_F\}\\
            &= \argmin_{\bm U \in \Z_{[a,b]}^{M\times R}}\{\|\bm U - \bm W\|^2_F\}\\
            &= \argmin_{\bm U \in \Z_{[a,b]}^{M\times R}}\{\|\bm U - \bm W\|^2_{[a,b]} + \|\bm U - \bm A\|^2_a + \|\bm U - \bm B\|^2_b\}\\
            &= \argmin_{\bm U \in \Z^{M\times R}}\{\|\bm U - \bm W\|^2_{[a,b]} + \|\bm U - \bm A\|^2_a + \|\bm U - \bm B\|^2_b\}\\
            &= \argmin_{\bm U \in \Z^{M\times R}}\{\|\bm U - \clamp_{[\alpha,\beta]}(\bm W)\|^2_F\}\\
            &= \round(\clamp_{[\alpha,\beta]}(\bm W)).
        \end{align*}
        The first equality is due to the definition of $\prox$ which is equivalent to the second equality. 
        In the third equality the matrices $\bm A\in\R^{M\times R}$ and $\bm B\in\R^{M\times R}$ have elements all equal to $a$ and $b$, respectively.
        The third equality is due to the fact that replacing $\|\bm U - \bm W\|^2_a + \|\bm U - \bm W\|^2_b$ with $\|\bm U - \bm A\|^2_a + \|\bm U - \bm B\|^2_b$ has no effect on the solution of the minimization. The fourth equality is also trivial due to the involved norms in the third equality. The fifth equality can be easily confirmed by the definition of $\clamp_{[\alpha,\beta]}$. Finally, in the last equality \eqref{eq:equivalence_prox_Tz} is invoked. A similar proof can be trivially followed for $\prox^f_{d_k}(\bm Z) = \round(\clamp_{[\alpha,\beta]}(\bm Z))$ as well.
    \end{proof}
    Now that the equivalence of iterates \eqref{eq:palm_updates} with the simple and closed-form steps in Algorithm \ref{alg: bcd for imf} is fully established, and the assumptions required for the convergence are verified to be met by problems \eqref{eq:IMF_surrogate} and \eqref{eq: imf problem} in proposition \ref{prop:assumptions}, proposition \ref{prop:convergence} can be travially invoked to establish the convergence of Algorithm \ref{alg: bcd for imf} to a locally optimal point of problem \eqref{eq: imf problem}.
\end{proof}